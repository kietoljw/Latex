% 指导教师给分及评阅意见
%请在“{}”中填入相应的内容:
%-------------------------------
% 开题意见
\newcommand{\kaitiYN}{同意开题}%也可以填入其它的意见
%===============================
% 评阅日期
\newcommand{\pingyueriqi}{\CJKfamily{kai}\nian 年4月26日}% 请在左边的{}中填入实际的月份和日期
%学习态度
\newcommand{\taidu}{}% 请在左边的{}中填入分数
%设计与撰写水平
\newcommand{\zhuanxie}{}% 请在左边的{}中填入分数
%计算机应用与规范
\newcommand{\jisuanji}{}% 请在左边的{}中填入分数
%科研能力
\newcommand{\keyan}{}% 请在左边的{}中填入分数
%文献整理与分析
\newcommand{\wenxian}{}% 请在左边的{}中填入分数
%研究结果的价值
\newcommand{\jieguo}{}% 请在左边的{}中填入分数
% 评定成绩
\newcommand{\pingdingdengji}{}% 请在左边的{}中填入上面各项分数之和
% 评阅意见
\newcommand{\zcomments}{
	%具体评阅意见在下面填写
	%评阅意见及下面各栏评分项目的得分都在文件夹~{\tt filesforteachers} 中的文件~{\tt zhidaojiaoshigeifen.tex} 内填写。
	%开题报告中的\textcolor{red}{导师意见}也在前述文件内填写(默认意见是:同意开题)。
	\minzi 同学的论文《\biaoti 》立题妥当,描述正确,论述清楚,有一定的实践意义。\\
	本文结构合理,层次分明,思路清晰,显示出作者具有一定的综合运用所学知识的能力,
	符合本科毕业论文的要求,同意其参加论文答辩。见及周边
	%-----------------------------------------------------------------------------
	% 选题(20)
	%\newcommand{\xuanti}{}% 请在左边的{}中填入分数
	%% 能力与态度(40)
	%\newcommand{\nengliyutaidu}{}% 请在左边的{}中填入分数
	%% 质量水平(40)
	%\newcommand{\zhiliangsuiping}{}% 请在左边的{}中填入分数
	%% 论文评定等级
	%\newcommand{\pingdingdengji}{}% 请在左边的{}中填入等级:优、良、合格、不合格
	%% 指导教师评阅分数
	%\newcommand{\jiaoshigeifen}{}
	%% 指导教师评阅分数×0.3
	%\newcommand{\jiaoshiR}{}% ×0.3
	%% 是否同意参加答辩
	%\newcommand{\dabianyn}{同意参加答辩}
	% 指导教师姓名
	%\newcommand{\jiaoshi}{包小敏}
}
