% 交叉评阅教师给分及评阅意见
% 请在“{}”中填入相应的内容:
%------------------------------
% 评阅日期
\newcommand{\jpingyueriqi}{\CJKfamily{kai}\nian 年4月28日}% 请在左边的{}中填入实际的月份和日期
% 选题价值(20)
\newcommand{\jxuanti}{}% 请在左边的{}中填入分数
%计算机应用与规范(15)
\newcommand{\jjisuanji}{}% 请在左边的{}中填入分数
%设计与撰写水平(40)
\newcommand{\jzhuanxie}{}% 请在左边的{}中填入分数
%语言表达(10)
\newcommand{\biaoda}{}% 请在左边的{}中填入分数
%文献整理与分析(15)
\newcommand{\jwenxian}{}% 请在左边的{}中填入分数
% 评定成绩
\newcommand{\jpingdingdengji}{}% 请在左边的{}中填入上面各项分数之和
% 交叉评阅教师姓名
\newcommand{\jiaocha}{交叉评阅教师姓名}% 请在左边{}内填入交叉评阅教师的姓名
%评阅意见
\newcommand{\jcomments}{
	%具体评阅意见在下面填写
	\jiaocha 及交叉评阅意见及下面各栏评分项目的得分都在文件夹~{\tt filesforteachers} 中的文件~{\tt jiaochageifen.tex} 内填写。
	%本论文在立题、阐述过程和所涉知识方面基本符合本科毕业论文的要求,基本观点、所涉知识无错误,
	%完全同意指导教师的评语和给定的成绩。
	%------------------------------------------------------------------------------------------
	%% 能力与态度(40)
	%\newcommand{\jnengliyutaidu}{}% 请在左边的{}中填入分数
	%% 质量水平(40)
	%\newcommand{\jzhiliangsuiping}{}% 请在左边的{}中填入分数
	%% 论文评定等级
	%\newcommand{\jpingdingdengji}{}% 请在左边的{}中填入等级:优、良、合格、不合格
	%% 选题指导思想(10)
	%\newcommand{\sixiang}{}
	%% 选题价值(10)
	%\newcommand{\jiazhi}{}
	%% 选题难度(5)
	%\newcommand{\nandu}{}
	%% 综合运用知识能力(10)
	%\newcommand{\nengli}{}
	%% 文献资料整理与分析能力()
	%\newcommand{\zhengli}{}
	%% 外文运用能力(5)
	%\newcommand{\yingwen}{}
	%% 语言运用能力(5)
	%\newcommand{\yuyan}{}
	%% 计算机运用能力(5)
	%\newcommand{\computer}{}
	%% 毕业论文撰写水平(30)
	%\newcommand{\xiezuo}{}
	%% 规范化程度(10)
	%\newcommand{\guifandu}{}
	%% 交叉评阅分数
	%\newcommand{\jiaochageifen}{}
	%% 交叉评阅分数×0.3
	%\newcommand{\jiaochaR}{}% ×0.3
}
