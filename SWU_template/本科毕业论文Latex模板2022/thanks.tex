%%%%%%%%%%%%%%%%%%%%%%%%%%%%%%%%%%%%%%%%%%%%%%%%%%%%%
\renewcommand{\baselinestretch}{\beishu}\normalsize %
%%%%%%%%%%%%%%%%%%%%%%%%%%%%%%%%%%%%%%%%%%%%%%%%%%%%%
%\noindent\large\zihao{-4}{\bf 致谢:}
\section*{致谢:}%\label{endofThesis}
\addcontentsline{toc}{section}{致谢}
%-----------------------------------
本模板目前的版本是由2006年完成的初始版经多年反复修改而来。西南大学数学与统计学院前后几十位同学在使用过程中所反馈的意见
是本模板发展的强大推动力。在此我要感谢所有使用过模板的同学,尤其是模板完成后最初几年使用过的同学,他们承受了初期模板中存在
的问题所带给他们的种种不便,有时甚至可能是痛苦。同时我也要感谢数学与统计学院的领导和某些老师对使用本模板的宽容态度,
正是由于他们的宽容,才使得本模板的使用和推广成为可能。我还要特别感谢我的同事彭作祥教授多年来的热情鼓励以及一些有益的建议,
他在2014年将模板推荐给统计系$2015$届的同学使用,从而使更多的同学开始接触到\LaTeX 和\CTeX 。自2016年以来,学院就``建议所有学生的毕业论文(设计)均使用latex模板排版"\citeu{tongzhi},
我本人当然欢迎更多的同学使用我这个模板。

希望大家在使用~\CTeX 编辑完成论文后能体会到当年~D.E.Knuth 教授在修订他的长篇巨著《The Arts of Computer Programming》\cite{knuth_2} 时还不得不抽出大量的时间来设计、
研制\LaTeX 的前身\TeX 的心情,以及\LaTeX 对当今社会的影响。同时我更希望~\CTeX 能给大家今后的工作、学习和生活带来更多的便利。

由于~\LaTeX 一直在不断的更新,同时计算机的操作系统也在不断升级,再加上本人也是一个业余的~\LaTeX 爱好者,水平有限,因此模板难免会有这样或那样的问题。请将问题或者建议~email 告诉我,我的~email
地址是~\href{mailto:xbao@swu.edu.cn}{\tt xbao@swu.edu.cn}。

\vspace{2cm}

\begin{center}
    \begin{tabular}{cp{3cm}l}
         &   & {\minzi}   \\
         &   & {\tijiaoriqi~\Printtime}   \\
         &   & {于\university~25 教~1520 } \\
     \end{tabular}
\end{center}
%\label{endofThesis}
