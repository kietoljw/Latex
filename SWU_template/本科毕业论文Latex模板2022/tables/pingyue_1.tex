\pagestyle{empty}
\section*{}\label{table:pingyue}
%\addcontentsline{toc}{section}{指导教师评阅表}
{\zihao{5}
	\begin{tabular}{|c|c|c|c|c|c|c|}
		\multicolumn{7}{c}{\includegraphics[height=0.8cm,angle=0]{preample/xishi}}   \\
		\multicolumn{7}{c}{\zihao{-2}{\textbf{本科毕业论文(设计)指导教师评阅表}}}\\\hline
		\multicolumn{2}{|c|}{论文(设计)题目}   & \multicolumn{5}{c|}{\biaoti\cobiaoti}                                        \\\hline
		\multicolumn{2}{|c|}{学生姓名}       & \multicolumn{2}{c|}{\minzi}   & 学号     & \multicolumn{2}{c|}{\xuehao}      \\\hline
		\multicolumn{2}{|c|}{学院、专业年级} & \multicolumn{5}{c|}{\school?\zhuanye 专业~\nianji}                          \\\hline
		\multicolumn{2}{|c|}{评阅人}         & \multicolumn{2}{c|}{\jiaoshi} & 评阅时间 & \multicolumn{2}{c|}{\pingyueriqi} \\\hline
		\multicolumn{1}{|p{.3cm}|}{\vspace{4.3cm}评阅意见} &
		\multicolumn{6}{c|}{
			\parbox[t][11cm][c]{14cm}{
				%具体评阅意见在此引入
				\zcomments
				%--------------------
			}
		} \\\hline
		\multicolumn{7}{|c|}{}\\[-8pt]
		\multicolumn{7}{|c|}{\raisebox{1ex}[0pt]{\zihao{4}{成绩评定}}}   \\\hline
		\multicolumn{2}{|c|}{评分项目}     & 得分   & 评分项目           & 得分      & 评分项目             & 得分         \\\hline
		%   \multicolumn{2}{|c|}{选题(20)}    & {\xuanti}  & 能力与态度(40) & {\nengliyutaidu} & 质量水平(40)& {\zhiliangsuiping} \\\hline
		\multicolumn{2}{|c|}{学习态度(15)} & \taidu & 设计或撰写水平(40) & \zhuanxie & 计算机应用与规范(10) & \jisuanji\\\hline
		\multicolumn{2}{|c|}{科研能力(15)} & \keyan & 文献整理与分析(10) & \wenxian  & 研究结果的价值(10)   & \jieguo  \\\hline
		\multicolumn{2}{|c|}{评定成绩}     & \multicolumn{5}{c|}{\pingdingdengji} \\\hline
		\multicolumn{2}{|c|}{评阅人签名}   & \multicolumn{5}{c|}{\jiaoshi} \\\hline
		\multicolumn{2}{|c|}{\raisebox{-1.8ex}{备注}} & \multicolumn{5}{p{11.6cm}|}{评阅意见主要从工作表现、能力水平、设计或论文质量三方面评价,
			各评分项目内涵见``指导教师评阅论文评定成绩标准"。} \\\hline
		\multicolumn{7}{c}{最终论文成绩=指导老师评阅成绩$\times 30\%$+交叉评阅成绩$\times 30\%$+答辩评定成绩$\times 40\%$}\\
		%\multicolumn{7}{c}{注:优(90分以上);良(80~89);中(70~79);及格(60 ~69);不及格(60以下)}\\
	\end{tabular}}