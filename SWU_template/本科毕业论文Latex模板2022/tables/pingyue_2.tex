\pagestyle{empty}
\section*{}\label{table:jiaochapingyue}
%\addcontentsline{toc}{section}{交叉评阅表}
{\zihao{5}
	\begin{tabular}{|c|c|c|c|c|c|c|}
		\multicolumn{7}{c}{\includegraphics[height=0.8cm,angle=0]{preample/xishi}}\\
		\multicolumn{7}{c}{\zihao{-2}{\textbf{本科毕业论文(设计)交叉评阅表}}} \\\hline
		\multicolumn{2}{|c|}{论文(设计)题目}    & \multicolumn{5}{l|}{\hfill{\biaoti\cobiaoti\hfill}}                          \\\hline
		\multicolumn{2}{|c|}{学生姓名}        & \multicolumn{2}{c|}{\minzi}   & 学号     & \multicolumn{2}{c|}{\xuehao}      \\\hline
		\multicolumn{2}{|c|}{学院、专业年级}  & \multicolumn{5}{c|}{\school、\zhuanye 专业~\nianji}                          \\\hline
		\multicolumn{2}{|c|}{评阅人}          & \multicolumn{2}{c|}{\jiaocha} & 评阅时间 & \multicolumn{2}{c|}{\jpingyueriqi}\\\hline
		\multicolumn{1}{|p{.3cm}|}{\vspace{4.2cm}评阅意见} & \multicolumn{6}{c|}{
			\parbox[t][11cm][c]{14cm}{
				%具体评阅意见在此引入
				\jcomments
				%---------------------
			}
		} \\\hline
		\multicolumn{7}{|c|}{}\\[-8pt]
		\multicolumn{7}{|c|}{\raisebox{1ex}[0pt]{\zihao{4}{成绩评定}}}   \\\hline
		\multicolumn{2}{|c|}{评分项目}     & 得分     & 评分项目             & 得分      & 评分项目            & 得分          \\\hline
		%\multicolumn{2}{|c|}{选题(20)}    & \jxuanti & 能力与态度(40) & \jnengliyutaidu & 质量水平(40)& \jzhiliangsuiping \\\hline
		\multicolumn{2}{|c|}{选题价值(20)} & \jxuanti & 计算机应用与规范(15) & \jjisuanji & 设计或撰写水平(40) & \jzhuanxie    \\\hline
		\multicolumn{2}{|c|}{语言表达(10)} & \biaoda  & 文献整理与分析(15)   & \jwenxian  &                    &               \\\hline
		\multicolumn{2}{|c|}{评定成绩}     & \multicolumn{5}{c|}{\jpingdingdengji}                                             \\\hline
		\multicolumn{2}{|c|}{评阅人签名}   & \multicolumn{5}{c|}{ } \\\hline
		%\multicolumn{2}{|c|}{评分项目}                    & 项目得分 & 评分项目                                               & 项目得分  & 评分项目                                         & 项目得分  \\\hline
		%   \multicolumn{2}{|p{2.23cm}|}{选题指导思想(10)}    & \sixiang & \multicolumn{1}{p{2.8cm}|}{文献资料整理与分析能力(10)} & \zhengli  & \multicolumn{1}{p{2.5cm}|}{毕业论文撰写水平(30)} & \xiezuo   \\\hline
		%   \multicolumn{2}{|c|}{选题价值(10)}                & \jiazhi  & 外文运用能力(5)                                        & \yingwen  & 规范化程度(10)                                   & \guifandu \\\hline
		%   \multicolumn{2}{|c|}{选题难度(5)}                 & \nandu   & 语言运用能力(5)                                        & \yuyan    &                                                  &          \\\hline
		%   \multicolumn{2}{|p{2.23cm}|}{综合运用知识能力(10)}& \nengli  & \multicolumn{1}{p{2.8cm}|}{计算机运用能力(5)}          & \computer &                                                  &          \\\hline
		%   \multicolumn{2}{|c|}{\textbf{总分}}                   & \multicolumn{5}{c|}{\jiaochageifen}          \\\hline
		%   \multicolumn{2}{|c|}{评阅人签名}                  & \multicolumn{5}{l|}{ }                        \\\hline
		\multicolumn{2}{|c|}{\raisebox{-1.8ex}{备注}}     & \multicolumn{5}{p{11.8cm}|}{评语意见主要从选题质量、能力水平、设计或论文质量三方面评价,
			各评分项目内涵见``指导教师评阅论文评定成绩标准"。} \\\hline
		\multicolumn{7}{c}{最终论文成绩=指导老师评阅成绩$\times 30\%$+交叉评阅成绩$\times 30\%$+答辩评定成绩$\times 40\%$}\\
		%\multicolumn{7}{c}{注:优(90分以上);良(80~89);中(70~79);及格(60~69);不及格(60以下)}\\
	\end{tabular}}
