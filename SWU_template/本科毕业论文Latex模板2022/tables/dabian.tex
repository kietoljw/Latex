%\begin{comment}
\section*{}\label{table:dabianjilu}
%\addcontentsline{toc}{section}{答辩记录}
\pagestyle{empty}{\zihao{5}
	\begin{tabular}{|p{.3cm}|p{2.2cm}|p{5cm}|p{1cm}|p{4cm}|}
		\multicolumn{5}{c}{\includegraphics[height=0.8cm,angle=0]{preample/xishi}} \\
		\multicolumn{5}{c}{\zihao{-2}\textbf{本科毕业论文(设计)答辩记录}}    \\\hline
		\multicolumn{2}{|c|}{\hspace*{\fill}论文(设计)题目\hspace*{\fill}} & \multicolumn{3}{c|}{\biaoti\cobiaoti}  \\\hline
		\multicolumn{2}{|c|}{\hspace*{\fill}学生姓名\hspace*{\fill}} & \hspace*{\fill}\minzi\hspace*{\fill} & \hspace*{\fill}学号\hspace*{\fill} & \hspace*{\fill}\xuehao\hspace*{\fill} \\\hline
		\multicolumn{2}{|c|}{学院、专业年级}                         & \multicolumn{3}{c|}{\school、\zhuanye 专业~\nianji} \\\hline
		\multicolumn{1}{|p{.3cm}|}{\vspace{2.7cm} 答辩记录} & \multicolumn{4}{c|}{
			\parbox[t][8.3cm][c]{12.6cm}{
				%具体答辩记录在此引入
				%答辩记录
%具体答辩记录在下面填写
%----------------------
答辩记录在文件夹~{\tt biaoge} 中的文件~{\tt dabianjilu.tex} 内填写。
%\begin{enumerate}
%      \item 递推数列非线性向线性转化的例子是哪个?\\
%      ---没来得及回答
%      \item 什么是非线性递推数列?线性递推数列是怎么定义的?例子\\
%      答:非线性递推数列就是递推关系为分式型或者含有根号和次方的数列。形如$a_{n+1} = \frac{aa_n+b}{ca_n+d}$
%      类型的,线性递推数列就是含有$a_{n+1} = \lambda_1a_1+\lambda_2a_2+\cdots+\lambda_na_n$ 递推关系式的数列。一般类型有
%      $a_{n+1} = pa_n+q$.
%      \item 有哪些形容词可以修饰递推数列?\\
%      答:高阶。(可以用齐次非齐次、常系数变系数、几阶)
%      \item 你所写的关系式$a_{n+1} = pa_n+q$是几阶的? \\
%      答:一阶。
%\end{enumerate}

				%-------------------
		}} \\\hline
		\multicolumn{1}{|p{.3cm}|}{\vspace{1.1cm} 评审意见} & \multicolumn{4}{c|}{
			\parbox[t][5.6cm][c]{13.6cm}{
				%具体评审意见也可在此引入
				\dcomments
				%--------------------
		}} \\\hline
		%\multicolumn{2}{|c}{答辩得分:}          & \multicolumn{1}{l}{\dabian\;分×\,0.4 = \dabianR}  &  \multicolumn{2}{l|}{评阅人给分:\quad \jiaochageifen\;分×\,0.3 = \jiaochaR} \\
		%\multicolumn{2}{|c}{指导教师给分:}      & \multicolumn{3}{l|}{\jiaoshigeifen\;分×\,0.3 = \jiaoshiR}                                    \\
		%\multicolumn{5}{|c|}{}\\[-8pt]
		\multicolumn{2}{|c}{{答辩评定成绩:}}    &  \multicolumn{3}{l|}{{\dabiancj}}                \\\hline
		\multicolumn{2}{|c}{答辩小组组长签名}    & \multicolumn{3}{|c|}{\hspace{4cm}\nian\hspace{.2cm}年\hspace{.5cm} 月\hspace{.5cm} 日} \\\hline
		\multicolumn{2}{|c}{答辩委员会主席签名}  & \multicolumn{3}{|c|}{\hspace{4cm}\nian\hspace{.2cm}年\hspace{.5cm} 月\hspace{.5cm} 日} \\\hline
		\multicolumn{5}{p{14.8cm}}{{\textbf{说明}}: 评审意见应包括:论文写作在立题、阐述过程和所涉知识方面是否符合本科毕业论文的要求;答辩中对所提问题给予的回答是充分、不够充分或者无法回答;学生答辩是否认真,状态是否良好。}\\
		\multicolumn{5}{p{14.8cm}}{最终论文成绩=指导老师评阅成绩$\times 30\%$+交叉评阅成绩$\times 30\%$+答辩评定成绩$\times 40\%$。}
		%\multicolumn{5}{c}{最终论文成绩=指导老师评阅成绩$\times 30\%$+交叉评阅成绩$\times 30\%$+答辩评定成绩$\times 40\%$}\\
		%\multicolumn{5}{c}{注:优(90分以上);良(80~89);中(70~79);及格(60~69);不及格(60以下)}\\
	\end{tabular}%\label{endofThesis}
}
%\end{comment}
