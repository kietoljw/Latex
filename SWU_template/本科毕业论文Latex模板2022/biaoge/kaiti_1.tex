\pagestyle{empty}
%\setcounter{page}{\thepage - 1}
\section*{}\label{table:kaiti_1}
%\addcontentsline{toc}{section}{开题报告}
{\zihao{5}
	\begin{tabular}{|p{3cm}|p{5.1cm}|p{1.5cm}|p{4cm}|}
		\multicolumn{4}{c}{\zihao{-2}\textbf{西南大学本科毕业论文(设计)开题报告}}\\\hline
		\hspace*{\fill}论文(设计)题目\hspace*{\fill}     & \multicolumn{3}{c|}{\biaoti\cobiaoti}                                                                                                       \\\hline
		\hspace*{\fill}学生姓名\hspace*{\fill}     & \hspace*{\fill}\minzi\hspace*{\fill}             & \hspace*{\fill}学号\hspace*{\fill} & \hspace*{\fill}\xuehao\hspace*{\fill}              \\\hline
		学院、专业年级                             & \multicolumn{3}{c|}{\school、\zhuanye 专业~\nianji}                                                                                \\\hline
		\hspace*{\fill}指导教师\hspace*{\fill} & {\hspace*{\fill}\jiaoshi\hspace*{\fill}} & 开题日期                           & {\hspace*{\fill}\kaitiriqi\hspace*{\fill}} \\\hline
		\multicolumn{4}{|l|}{
			\parbox[t][9.2cm][s]{14.2cm}{
				1. 本课题研究意义:\\*[.2cm]
				具体内容在文件夹~{\tt biaoge} 中的文件~{\tt kaiti\_1.tex} 内填写。
				%----------------
		}} \\\hline
		\multicolumn{4}{|l|}{
			\parbox[t][9.2cm][t]{14.2cm}{
				2. 研究内容:\\*[.2cm]
				填写位置同上。
				%----------------
		}} \\\hline
\end{tabular}}
