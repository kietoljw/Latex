%%%%%%%%%%%%%%%%%%%%%%%%%%%%%%%%%%%%%%%%
% 文件名 :     myTemplate.tex          %
%                                      %
% 作者:        包小敏                  %
%                                      %
% 单位:        西南大学数学与统计学院  %\alert{text}
%                                      %
% 创建于:      2011年02月21日          %
% 最后修改于:  2019年11月28日          %
%%%%%%%%%%%%%%%%%%%%%%%%%%%%%%%%%%%%%%%%
\documentclass[x11names,a4paper,AutoFakeBold]{ctexart}
%======================Include Packages========================
\usepackage{SWUthesis}
\usepackage{makecell}
\usepackage{xcolor}
\usepackage{cleveref}%聪明的引用
\usepackage{caption}
\usepackage{bicaption}
\usepackage{threeparttable}
\usepackage{wrapfig}%浮动图片
\usepackage{tasks}%横向列表排版
\usepackage{diagbox}%制作斜线表头
\usepackage{appendix}
%\usepackage{tabularray}

\usepackage{background}
\definecolor{grey}{rgb}{0.91,0.91,0.91}
\SetBgContents{LJW}
\SetBgAngle{-45}
\SetBgColor{grey}
\SetBgOpacity{0.08}
%定义字体
\setCJKfamilyfont{kai}{simkai.ttf}
\crefname{theorem}{定理}{定理} %定义引用标签
\crefname{lemma}{引理}{引理}
\crefname{definition}{定义}{定义}
\crefname{figure}{图}{图}
\crefname{table}{表}{表}
\crefname{algorithm}{算法}{算法}
\renewcommand{\thetable}{\arabic{table}}%改变标题为罗马标签
\renewcommand{\thefigure}{\arabic{figure}}
%设置标题标签和文本字体格式
\captionsetup[table]{labelfont={bf,small },textfont={bf,small},justification=centering}
\captionsetup[figure]{labelfont={bf,small},textfont={bf,small},justification=centering}
\captionsetup[figure][bi-second]{name=Figure} %设置图的英文编号前缀
\captionsetup[table][bi-second]{name=Table} %设置表的英文编号前缀

%\fancyhf{}
\pagestyle{fancy}
\fancyhead[L]{}
\fancyhead[C]{}
\fancyhead[R]{}
\fancyfoot[L]{}
\fancyfoot[C]{\zihao{-5}\small 第\;\thepage\;页\;共\;\pageref*{LastPage}\;页}
\fancyfoot[R]{\tiny\color{grey}LJW}

%%%%%%%%%%%%%%%%%%%%%%%%%%%%%%%%%%%%%%%%%%%%%%%%%%%%%%%%%%%%%%%%
% 指导教师给分及评阅意见
%请在“{}”中填入相应的内容:
%-------------------------------
% 开题意见
\newcommand{\kaitiYN}{同意开题}%也可以填入其它的意见
%===============================
% 评阅日期
\newcommand{\pingyueriqi}{\CJKfamily{kai}\nian 年4月26日}% 请在左边的{}中填入实际的月份和日期
%学习态度
\newcommand{\taidu}{}% 请在左边的{}中填入分数
%设计与撰写水平
\newcommand{\zhuanxie}{}% 请在左边的{}中填入分数
%计算机应用与规范
\newcommand{\jisuanji}{}% 请在左边的{}中填入分数
%科研能力
\newcommand{\keyan}{}% 请在左边的{}中填入分数
%文献整理与分析
\newcommand{\wenxian}{}% 请在左边的{}中填入分数
%研究结果的价值
\newcommand{\jieguo}{}% 请在左边的{}中填入分数
% 评定成绩
\newcommand{\pingdingdengji}{}% 请在左边的{}中填入上面各项分数之和
% 评阅意见
\newcommand{\zcomments}{
	%具体评阅意见在下面填写
	%评阅意见及下面各栏评分项目的得分都在文件夹~{\tt filesforteachers} 中的文件~{\tt zhidaojiaoshigeifen.tex} 内填写。
	%开题报告中的\textcolor{red}{导师意见}也在前述文件内填写(默认意见是:同意开题)。
	\minzi 同学的论文《\biaoti 》立题妥当,描述正确,论述清楚,有一定的实践意义。\\
	本文结构合理,层次分明,思路清晰,显示出作者具有一定的综合运用所学知识的能力,
	符合本科毕业论文的要求,同意其参加论文答辩。见及周边
	%-----------------------------------------------------------------------------
	% 选题(20)
	%\newcommand{\xuanti}{}% 请在左边的{}中填入分数
	%% 能力与态度(40)
	%\newcommand{\nengliyutaidu}{}% 请在左边的{}中填入分数
	%% 质量水平(40)
	%\newcommand{\zhiliangsuiping}{}% 请在左边的{}中填入分数
	%% 论文评定等级
	%\newcommand{\pingdingdengji}{}% 请在左边的{}中填入等级:优、良、合格、不合格
	%% 指导教师评阅分数
	%\newcommand{\jiaoshigeifen}{}
	%% 指导教师评阅分数×0.3
	%\newcommand{\jiaoshiR}{}% ×0.3
	%% 是否同意参加答辩
	%\newcommand{\dabianyn}{同意参加答辩}
	% 指导教师姓名
	%\newcommand{\jiaoshi}{包小敏}
}
% 指导教师评阅及打分
% 交叉评阅教师给分及评阅意见
% 请在“{}”中填入相应的内容:
%------------------------------
% 评阅日期
\newcommand{\jpingyueriqi}{\CJKfamily{kai}\nian 年4月28日}% 请在左边的{}中填入实际的月份和日期
% 选题价值(20)
\newcommand{\jxuanti}{}% 请在左边的{}中填入分数
%计算机应用与规范(15)
\newcommand{\jjisuanji}{}% 请在左边的{}中填入分数
%设计与撰写水平(40)
\newcommand{\jzhuanxie}{}% 请在左边的{}中填入分数
%语言表达(10)
\newcommand{\biaoda}{}% 请在左边的{}中填入分数
%文献整理与分析(15)
\newcommand{\jwenxian}{}% 请在左边的{}中填入分数
% 评定成绩
\newcommand{\jpingdingdengji}{}% 请在左边的{}中填入上面各项分数之和
% 交叉评阅教师姓名
\newcommand{\jiaocha}{交叉评阅教师姓名}% 请在左边{}内填入交叉评阅教师的姓名
%评阅意见
\newcommand{\jcomments}{
	%具体评阅意见在下面填写
	\jiaocha 及交叉评阅意见及下面各栏评分项目的得分都在文件夹~{\tt filesforteachers} 中的文件~{\tt jiaochageifen.tex} 内填写。
	%本论文在立题、阐述过程和所涉知识方面基本符合本科毕业论文的要求,基本观点、所涉知识无错误,
	%完全同意指导教师的评语和给定的成绩。
	%------------------------------------------------------------------------------------------
	%% 能力与态度(40)
	%\newcommand{\jnengliyutaidu}{}% 请在左边的{}中填入分数
	%% 质量水平(40)
	%\newcommand{\jzhiliangsuiping}{}% 请在左边的{}中填入分数
	%% 论文评定等级
	%\newcommand{\jpingdingdengji}{}% 请在左边的{}中填入等级:优、良、合格、不合格
	%% 选题指导思想(10)
	%\newcommand{\sixiang}{}
	%% 选题价值(10)
	%\newcommand{\jiazhi}{}
	%% 选题难度(5)
	%\newcommand{\nandu}{}
	%% 综合运用知识能力(10)
	%\newcommand{\nengli}{}
	%% 文献资料整理与分析能力()
	%\newcommand{\zhengli}{}
	%% 外文运用能力(5)
	%\newcommand{\yingwen}{}
	%% 语言运用能力(5)
	%\newcommand{\yuyan}{}
	%% 计算机运用能力(5)
	%\newcommand{\computer}{}
	%% 毕业论文撰写水平(30)
	%\newcommand{\xiezuo}{}
	%% 规范化程度(10)
	%\newcommand{\guifandu}{}
	%% 交叉评阅分数
	%\newcommand{\jiaochageifen}{}
	%% 交叉评阅分数×0.3
	%\newcommand{\jiaochaR}{}% ×0.3
}
      % 交叉评阅及打分
% 答辩分数、等级及评审意见
% 请在“{}”中填入相应的内容:
%------------------------------
%% 答辩分数
%\newcommand{\dabian}{}
%% 答辩分数×0.4
%\newcommand{\dabianR}{}  % ×0.4
%% 论文成绩
\newcommand{\thesisGrade}{合格} % 论文成绩
% 论文评定等级
\newcommand{\dabiancj}{88}    % 答辩成绩
% 答辩评审意见
\newcommand{\dcomments}{
	% 具体评审意见在下面填写
	评审意见及下面的论文评定等级都在文件夹~{\tt filesforteachers} 中的文件~{\tt dabianchengji.tex} 内填写。
	%\minzi 同学的论文《\biaoti 》立题合理,具有一定的实践意义,所涉知识符合本科毕业论文的要求。
	%在答辩中叙述清楚,有条理,能正确回答所提问题,答辩态度认真,状态良好。
	%----------------------
}
      % 答辩成绩
%%%%%%%%%%%%%%%%%%%%%%%%%%%%%%%%%%%%%%%%%%%%%%%%%%%%%%%%%%%%%%%%
\makeindex%生成索引
%===========================请填入相应的内容===================%
\newcommand{\university}{西南大学}                             %
\newcommand{\school}    {数学与统计学院}                       %
\newcommand{\city}      {重庆 400715}                          %
\newcommand{\biaoti}{毕业论文模板}                 % 中文论文题目。如题目太长,可把一部分放在副标题的位置
\newcommand{\cobiaoti}{附带说明}                  % 副标题,没有的话将“---附带简单的使用说明”删除
\newcommand{\enbiaoti}{Undergraduate Thesis {\LaTeX} Template} % 英文论文题目
\newcommand{\nianji}  {2018 级}                                % 年级
\newcommand{\zhuanye}{统计学}                  % 专业
\newcommand{\minzi}{{\CJKfamily{kai}李嘉伟}}                   % 学生中文姓名
\newcommand{\englishname}{Name of Student}                     % 学生英文姓名
\newcommand{\jiaoshi}{\CJKfamily{kai}李婷婷}                   % 指导教师姓名
\renewcommand{\jiaocha}{{\CJKfamily{kai}交叉评阅教师}}         % 请在左边{}内填入交叉评阅教师的姓名
\newcommand{\xuehao}{\CJKfamily{kai}222018314210017}           % 学号
\newcommand{\nian}{\CJKfamily{kai}2020}                        % 毕业年份
\newcommand{\nianxian}{十四}                                   % 已使用年数(=2019-2006)
\newcommand{\kaitiriqi}{\CJKfamily{kai}2019年12月20日}         % 开题日期
\newcommand{\tijiaoriqi}{\CJKfamily{kai}\today}                % 提交日期
\newcommand{\dayinriqi}{\CJKfamily{kai}\today}                 % 打印日期
%===============================================================
% 若修改过程中某个文件不需修改,则将\includeonly中这个文件名注释掉可以加快编译速度。
% 请用PDFLATEX编译,直接生成PDF文件。也可用 LATEX 编译,但速度可能会非常慢,有时甚至可能会出错。
%===============================================================
\includeonly{tables/coverpage,
             tables/pingyue_1,
             tables/pingyue_2,
             tables/dabian,
             biaoge/kaiti_1,
             biaoge/kaiti_2,
             fulu
}
%======================
\makeindex

\begin{document}
\begin{titlepage}
\begin{center}
	%要学校的logo时使用此封面头
	%    \begin{tabular}{cc}
		%        \includegraphics[width=3cm]{preample/xishilogo.eps}
		%        &
		%        \raisebox{8.6ex}[0pt]{
			%        \begin{tabular}{c}
				%        \includegraphics[height=1.3cm,angle=0]{preample/xishi.eps}\\\\
				%
				%        {\textbf{\zihao{1} 本科毕业论文(设计)}}
				%    \end{tabular}}\\
		%\end{tabular}
		%不要学校的logo时使用此封面头
		\begin{tabular}{c}
			\includegraphics[height=1.3cm,angle=0]{preample/xishi.eps}\\\\
			
			\hspace{.8cm}{\textbf{\zihao{1} 本科毕业论文(设计)}}
		\end{tabular}
		%-------------------------------------------------
	\end{center}
	\vspace{2.0cm}
	
	\huge\zihao{2}
	\begin{tabular}{ll}
		{\textbf{题\quad 目}}& \biaoti \\\cline{2-2}
		&\cobiaoti\\
	\end{tabular}
	\vspace{2.4cm}
	
	\begin{center}
		\begin{tabular}{rrllc}
			\huge\zihao{3}{\textbf{学}} & & & \huge\zihao{3}{\textbf{院}} & {\huge\zihao{3}{\CJKfamily{kai}\school}}\\\cline{5-5}
			\huge\zihao{3}{\textbf{专}} & & & \huge\zihao{3}{\textbf{业}} & {\huge\zihao{3}{\CJKfamily{kai}\zhuanye}}\\\cline{5-5}
			\huge\zihao{3}{\textbf{年}} & & & \huge\zihao{3}{\textbf{级}} & {\huge\zihao{3}\nianji}\\\cline{5-5}
			\huge\zihao{3}{\textbf{学}} & & & \huge\zihao{3}{\textbf{号}} & {\huge\zihao{3}\xuehao}\\\cline{5-5}
			\huge\zihao{3}{\textbf{姓}} & & & \huge\zihao{3}{\textbf{名}} & {\huge\zihao{3}\minzi}\\\cline{5-5}
			\huge\zihao{3}{\textbf{指}} & \huge\zihao{3}{\textbf{导}} & \huge\zihao{3}{\textbf{教}}&\huge\zihao{3}{\textbf{师}}& {\huge\zihao{3}\jiaoshi}\\\cline{5-5}
			\huge\zihao{3}{\textbf{成}} & & & \huge\zihao{3}{\textbf{绩}} & {\huge\zihao{3}{\CJKfamily{kai}\thesisGrade}}\\\cline{5-5}
		\end{tabular}
		\vspace{3.2cm}
		
		{\LARGE\zihao{3}\dayinriqi}
	\end{center}

\end{titlepage}
%====================生=成=目=录===================================
%%%%%%%%%%%%%%%%%%%%%%
\pagenumbering{roman}%
\thispagestyle{plain}%
%%%%%%%%%%%%%%%%%%%%%%
\tableofcontents
\thispagestyle{plain}
\newpage
%==========================摘要===================================
%%%%%%%%%%%%%%%%%%%%%%%%
\pagenumbering{arabic} %
\setcounter{page}{1}   %
%%%%%%%%%%%%%%%%%%%%%%%%
\begin{center}
{\heiti\bf\LARGE\zihao{3}{\biaoti\cobiaoti}}

\vspace{0.3cm}
{\zihao{-4}\minzi}

{\zihao{5}\university\school,\city}
\end{center}
\addcontentsline{toc}{section}{摘要}
\begin{center}
\begin{minipage}[t]{13cm}
\noindent{\zihao{5}{\bf 摘要:}
\fangsong
%%%%%%%%%%%%%%%%%%%%%%%%%%%%%%%%%%%%%%%%%%%%%%%%%%%%%%%%%%%%%%%%%%%%%%%%%%%%%%%%%%%%%%%%//
% 请填入中文摘要:
本模版是为{\university\school}本科毕业生而设计的,
模版根据{\school}论文的Word模版中所描述的要求编写。目的是简化和
规范学位论文的撰写,使得论文作者可以将精力集中到论文的内容上而不是浪费在版面设置上。
使用者只需了解{\LaTeX}的基本概念即可使用本模版。

}\vspace{0.1cm}

\noindent{\songti\zihao{5}{\bf 关键词:} \LaTeX ;模版;论文;版面;参数
%%%%%%%%%%%%%%%%%%%%%%%%%%%%%%%%%%%%%%%%%%%%%%%%%%%%%%%%%%%%%%%%%%%%%%%%%%%%%%%%%%%%%%%%%%
}\vspace{0.6cm}
\end{minipage}
\end{center}
\begin{center}
{\Large\bf\zihao{4}\enbiaoti}

\vspace{0.3cm}
{\normalsize\zihao{5}\englishname}

{\small School of Mathematics \& Statistics, Southwest University,Chongqing 400715}
\end{center}
\addcontentsline{toc}{section}{Abstract}
\noindent{\normalsize\zihao{5}{\bf Abstract:}
%%%%%%%%%%%%%%%%%%%%%%%%%%%%%%%%%%%%%%%%%%%%%%%%%%%%%%%%%%%%%%%%%%%%%%%%%%%%%%%%%%%%%%%%%//
% 请填入英文摘要:
This {\LaTeX} template is designed for the undergraduate students of
the School of Mathematics and Statistics at the Southwest
University. The design of the template followed the undergraduates
thesis requirements described in the Word templates. Our purposes
are to reduce the editing workload for the students, standardize the
layout of their theses. Only basic knowledge of the {\LaTeX} is
required to use this template.

}\vspace{0.1cm}

\noindent{\normalsize\zihao{5}{\bf Key words:} {\LaTeX};template;thesis;layout;parameters}
%\end{minipage}
%\end{center}

%=======================正文开始==================================
\input{body}
%=======================参考文献=================================%
\renewcommand{\baselinestretch}{\yuanbeishu} %
\normalsize\zihao{5}                         %
%\vspace{1cm}                                %
\addcontentsline{toc}{section}{参考文献}     %
\begin{thebibliography}{9}                   %
%\vspace{.4cm}
%%%%%%%%%%%%%%%%%%%%%%%%%%%%%%%%%%%%%%%%%%%%%%%%%%%%%%%%%%%%%%%%%%
\bibitem{moban_1}
    西南大学数学与统计学院.
    \textsl{本科毕业论文(设计)规范化要求(论文最新模版)}.
\bibitem{tongzhi}
    西南大学数学与统计学院.
    \textsl{关于2017届本科毕业论文工作安排的通知}.
\bibitem{experqc}
    C.H.Bennett, F.Bessette, G.Brassard, et al,
    \textsl{Experimental quantum cryptography} [J].
    Journal of Cryptology, Vol.5, No.1(1992), 3--28.
\bibitem{latex}
    \TeX\, Guru.
    \textsl{\LaTeX 2e 用户手册}. 1999.
\bibitem{knuth}
    Donald E. Knuth,
    \textsl{The \TeX book} [M]. Addison-Wesley, 1984.
\bibitem{knuth_2}
    Donald E. Knuth,
    \textsl{The Arts of Computer Programming} [M]. 北京:机械工业出版社, 2008.
\bibitem{stinson}
    D. R. Stinson,
    \textsl{Cryptography---Theory and Practice (Third Edition)}[M]. Chapman \& Hall/CRC, Taylor \& Francis Group, Roca Raton, FL, USA. 164.
\bibitem{xiong}
    熊全淹,
    \textsl{近世代数} [M]. 武汉大学出版社, 1995年.
\bibitem{companion}
    Frank Mittelbach, Michel Goossens.
    \textsl{The \LaTeX \,Companion (Second Edition)}[M]. Addison-Wesley, 2005.
\bibitem{zhanglibo}
    张林波等.
    \textsl{CCT中外文科技排版系统} [M].北京:海洋出版社,1993年。
\bibitem{zhang}
    张禾瑞,
    \textsl{近世代数} [M].高等教育出版社, 1978年.
\bibitem{das}
    Das,M.L., Saxena, A., and Phatak,D.B.
    \emph{Algorithms and Approaches of Proxy Signature: A Servey}. arXiv:cs/0612098v1 [cs.CR],20 Dec 2006. ~\href{https://arxiv.org/pdf/cs/0612098.pdf}{https://arxiv.org/pdf/cs/0612098.pdf.}

\end{thebibliography}

%=========================索引===================================%
\printindex %如不要索引请将此行和下行注释掉
\addcontentsline{toc}{section}{索引}
%=========================致谢===================================%
%%%%%%%%%%%%%%%%%%%%%%%%%%%%%%%%%%%%%%%%%%%%%%%%%%%%%
\renewcommand{\baselinestretch}{\beishu}\normalsize %
%%%%%%%%%%%%%%%%%%%%%%%%%%%%%%%%%%%%%%%%%%%%%%%%%%%%%
%\noindent\large\zihao{-4}{\bf 致谢:}
\section*{致谢:}%\label{endofThesis}
\addcontentsline{toc}{section}{致谢}
%-----------------------------------
本模板目前的版本是由2006年完成的初始版经多年反复修改而来。西南大学数学与统计学院前后几十位同学在使用过程中所反馈的意见
是本模板发展的强大推动力。在此我要感谢所有使用过模板的同学,尤其是模板完成后最初几年使用过的同学,他们承受了初期模板中存在
的问题所带给他们的种种不便,有时甚至可能是痛苦。同时我也要感谢数学与统计学院的领导和某些老师对使用本模板的宽容态度,
正是由于他们的宽容,才使得本模板的使用和推广成为可能。我还要特别感谢我的同事彭作祥教授多年来的热情鼓励以及一些有益的建议,
他在2014年将模板推荐给统计系$2015$届的同学使用,从而使更多的同学开始接触到\LaTeX 和\CTeX 。自2016年以来,学院就``建议所有学生的毕业论文(设计)均使用latex模板排版"\citeu{tongzhi},
我本人当然欢迎更多的同学使用我这个模板。

希望大家在使用~\CTeX 编辑完成论文后能体会到当年~D.E.Knuth 教授在修订他的长篇巨著《The Arts of Computer Programming》\cite{knuth_2} 时还不得不抽出大量的时间来设计、
研制\LaTeX 的前身\TeX 的心情,以及\LaTeX 对当今社会的影响。同时我更希望~\CTeX 能给大家今后的工作、学习和生活带来更多的便利。

由于~\LaTeX 一直在不断的更新,同时计算机的操作系统也在不断升级,再加上本人也是一个业余的~\LaTeX 爱好者,水平有限,因此模板难免会有这样或那样的问题。请将问题或者建议~email 告诉我,我的~email
地址是~\href{mailto:xbao@swu.edu.cn}{\tt xbao@swu.edu.cn}。

\vspace{2cm}

\begin{center}
    \begin{tabular}{cp{3cm}l}
         &   & {\minzi}   \\
         &   & {\tijiaoriqi~\Printtime}   \\
         &   & {于\university~25 教~1520 } \\
     \end{tabular}
\end{center}
%\label{endofThesis}

%=========================附录===================================%
\begin{appendix}
%\begin{appendix}
\section{Matlab 代码} \label{appendix:A}

下面是我写的一个~Matlab
代码,这个代码可以给出下面这个
1989年第30届国际数学奥林匹克(IMO)竞赛的一道试题的一个解答:\vspace{.5cm}

\noindent 求证:集合$\Set{1, 2,
\cdots,1989}$可以分为$117$个互不相交的子集
$$A_i, \qquad i = 1, 2, \cdots,117$$
使得
\begin{enumerate}
    \item 每个$A_i$都含有$17$个元素;
    \item 每个$A_i$中所有元素之和相同。
\end{enumerate}\vspace{.5cm}

这个代码的文件名为~{\tt equalsumpartition.m},运行时在~Matlab 命令窗口键入下面的命令
\begin{verbatim}
        equalsumpartition(1989,117)
\end{verbatim}
然后回车即可。
这里引入所用的命令是
\begin{verbatim}
       \lstinputlisting[language=Matlab]{codes/equalsumpartition.m}
\end{verbatim}
注意,因为文件~{\tt equalsumpartition.m} 存放在子文件夹~{\tt codes }中,所以引入时,文件名前面还必须加上路径名~{\tt codes/}。
%\textbf{\textcolor[rgb]{0.98,0.00,0.00}{ Input Matlab source:}}
\lstinputlisting[language=Matlab]{codes/equalsumpartition.m}
\section{Mathematica 代码}\label{appendix:B}
按第~\ref{sec:codeinludsion} 节介绍的方法直接引入~Mathematica 代码目前还有一些问题,一个临时解决办法是先将~Mathematica 代码另存为.txt 文件,然后用命令
\begin{verbatim}
        \lstinputlisting[language=Mathematica]{myfile.txt}
\end{verbatim}
引入。%注意,这里和第~\ref{sec:codeinludsion} 节介绍的方法的不同之处是没有了\verb|[language=代码语言]| 这一部分。
下面就是一个按这种方法引入的一个~Mathematica 代码:{\tt primeSieve.txt},这里用的命令是
\begin{verbatim}
       \lstinputlisting[language=Mathematica]{codes/primeSieve.txt}
\end{verbatim}
注意,文件~{\tt primeSieve.txt} 也存放在子文件夹~{\tt codes} 中。
这个代码实现了著名的~Eratosthenes 筛法。对一个任意给定的正整数$n$,通过~Eratosthenes 筛法,可以找出所有不超过$n$的素数。本代码首先在$[50,200]$中随机地选一个数作为$n$(如果要用其它的数,可通过调整代码中赋给的~{\tt m} 和~{\tt M} 的值来实现),然后给出所有不超过$n$的素数。如果要查看中间结果,只需将一个非$0$ 的值赋给变量$d$,然后运行程序即可。
\lstinputlisting[language=Mathematica]{codes/primeSieve.txt}
\section{R 代码}\label{appendix:C}
此例子由彭作祥教授提供,是$\mathbf{R}$软件下编写的程序,为寿险精算中的一个简单例子。计算
在每年末偿还金额中的利息和本金金额。这个代码的文件名为~{\tt
example.R},在$\mathbf{R}$中打开运行即可。引入命令是
\begin{verbatim}
       \lstinputlisting[language=R]{codes/example.R}
\end{verbatim}
\lstinputlisting[language=R]{codes/example.R}
\section{DES C++ 代码}\label{appendix:D}
下面是一个文件名为~{\tt mytestdes.cpp} 的~C++ 代码,引入命令是
\begin{verbatim}
       \lstinputlisting[language={[ANSI]C++}]{codes/mytestdes.cpp}
\end{verbatim}
这个代码实现了著名的对称密码算法~DES。
%\textbf{\textcolor[rgb]{0.98,0.00,0.00}{ Input Matlab source:}}
\lstinputlisting[language={[ANSI]C++}]{codes/mytestdes.cpp}
%\end{appendix}
%如没有附录,请将这三句和下面的一句“\setcounter{page}{\thepage - 1}”都注释掉
\end{appendix}
\setcounter{page}{\thepage - 1}
%%%%%%%%%%%%%%%%%%%%%%%%%%%%%%%%%%%%%%%%%%%%%%%%%%%%%%%%%%%%%%%%%%
\label{endofThesis}
%=======================开题报告一===============================%
\pagestyle{empty}
%\setcounter{page}{\thepage - 1}
\section*{}\label{table:kaiti_1}
%\addcontentsline{toc}{section}{开题报告}
{\zihao{5}
	\begin{tabular}{|p{3cm}|p{5.1cm}|p{1.5cm}|p{4cm}|}
		\multicolumn{4}{c}{\zihao{-2}\textbf{西南大学本科毕业论文(设计)开题报告}}\\\hline
		\hspace*{\fill}论文(设计)题目\hspace*{\fill}     & \multicolumn{3}{c|}{\biaoti\cobiaoti}                                                                                                       \\\hline
		\hspace*{\fill}学生姓名\hspace*{\fill}     & \hspace*{\fill}\minzi\hspace*{\fill}             & \hspace*{\fill}学号\hspace*{\fill} & \hspace*{\fill}\xuehao\hspace*{\fill}              \\\hline
		学院、专业年级                             & \multicolumn{3}{c|}{\school、\zhuanye 专业~\nianji}                                                                                \\\hline
		\hspace*{\fill}指导教师\hspace*{\fill} & {\hspace*{\fill}\jiaoshi\hspace*{\fill}} & 开题日期                           & {\hspace*{\fill}\kaitiriqi\hspace*{\fill}} \\\hline
		\multicolumn{4}{|l|}{
			\parbox[t][9.2cm][s]{14.2cm}{
				1. 本课题研究意义:\\*[.2cm]
				具体内容在文件夹~{\tt biaoge} 中的文件~{\tt kaiti\_1.tex} 内填写。
				%----------------
		}} \\\hline
		\multicolumn{4}{|l|}{
			\parbox[t][9.2cm][t]{14.2cm}{
				2. 研究内容:\\*[.2cm]
				填写位置同上。
				%----------------
		}} \\\hline
\end{tabular}}

%=======================开题报告二===============================%
\pagestyle{empty}
\section*{}\label{table:kaiti_2}
{\zihao{5}
	\begin{tabular}{|c|c|c|c|c|c|}\hline
		\multicolumn{6}{|c|}{
			\parbox[t][11cm][t]{14.6cm}{
				3. 技术路线、研究方法和进度:\\*[.2cm]
				具体内容在文件夹~{\tt biaoge} 中的文件~{\tt kaiti\_2.tex} 内填写。下面的\textcolor{red}{导师意见}在文件夹~{\tt filesforteachers} 中的文件~{\tt zhidaojiaoshigeifen.tex} 内填写(默认意见是:同意开题)。
		}}\\\hline
		\multicolumn{6}{|c|}{
			\parbox[t][.5cm][t]{14.6cm}{
				4. 导师意见:
		}}\\
		\multicolumn{6}{|c|}{
			\parbox[t][4cm][c]{14.6cm}{
				\centering {\Large \kaitiYN}%同意开题}
	}}\\
	\multicolumn{6}{|c|}{
		\parbox[t][.6cm][t]{14.6cm}{
			\hspace{5cm}指导教师(签名):{\jiaoshi}
	}}\\
	\multicolumn{6}{|c|}{
		\parbox[t][.6cm][t]{14.6cm}{
			\hspace{8cm}\kaitiriqi
	}}\\\hline
	\multicolumn{6}{|c|}{
		\parbox[t][3cm][t]{14.6cm}{
			5. 学院意见:
	}}\\
	\multicolumn{6}{|c|}{
		\parbox[t][.6cm][t]{14.6cm}{
			\hspace{6cm}学院(盖章):
	}}\\
	\multicolumn{6}{|c|}{
		\parbox[t][.6cm][t]{14.26cm}{
			\hspace{8cm} \hspace{.2cm}年\hspace{.5cm} 月\hspace{.5cm} 日
	}}\\\hline
\end{tabular}
}
%========================任务书==================================%
%\include{biaoge/renwu}
%=======================指导教师评阅表===========================%
\pagestyle{empty}
\section*{}\label{table:pingyue}
%\addcontentsline{toc}{section}{指导教师评阅表}
{\zihao{5}
	\begin{tabular}{|c|c|c|c|c|c|c|}
		\multicolumn{7}{c}{\includegraphics[height=0.8cm,angle=0]{preample/xishi}}   \\
		\multicolumn{7}{c}{\zihao{-2}{\textbf{本科毕业论文(设计)指导教师评阅表}}}\\\hline
		\multicolumn{2}{|c|}{论文(设计)题目}   & \multicolumn{5}{c|}{\biaoti\cobiaoti}                                        \\\hline
		\multicolumn{2}{|c|}{学生姓名}       & \multicolumn{2}{c|}{\minzi}   & 学号     & \multicolumn{2}{c|}{\xuehao}      \\\hline
		\multicolumn{2}{|c|}{学院、专业年级} & \multicolumn{5}{c|}{\school?\zhuanye 专业~\nianji}                          \\\hline
		\multicolumn{2}{|c|}{评阅人}         & \multicolumn{2}{c|}{\jiaoshi} & 评阅时间 & \multicolumn{2}{c|}{\pingyueriqi} \\\hline
		\multicolumn{1}{|p{.3cm}|}{\vspace{4.3cm}评阅意见} &
		\multicolumn{6}{c|}{
			\parbox[t][11cm][c]{14cm}{
				%具体评阅意见在此引入
				\zcomments
				%--------------------
			}
		} \\\hline
		\multicolumn{7}{|c|}{}\\[-8pt]
		\multicolumn{7}{|c|}{\raisebox{1ex}[0pt]{\zihao{4}{成绩评定}}}   \\\hline
		\multicolumn{2}{|c|}{评分项目}     & 得分   & 评分项目           & 得分      & 评分项目             & 得分         \\\hline
		%   \multicolumn{2}{|c|}{选题(20)}    & {\xuanti}  & 能力与态度(40) & {\nengliyutaidu} & 质量水平(40)& {\zhiliangsuiping} \\\hline
		\multicolumn{2}{|c|}{学习态度(15)} & \taidu & 设计或撰写水平(40) & \zhuanxie & 计算机应用与规范(10) & \jisuanji\\\hline
		\multicolumn{2}{|c|}{科研能力(15)} & \keyan & 文献整理与分析(10) & \wenxian  & 研究结果的价值(10)   & \jieguo  \\\hline
		\multicolumn{2}{|c|}{评定成绩}     & \multicolumn{5}{c|}{\pingdingdengji} \\\hline
		\multicolumn{2}{|c|}{评阅人签名}   & \multicolumn{5}{c|}{\jiaoshi} \\\hline
		\multicolumn{2}{|c|}{\raisebox{-1.8ex}{备注}} & \multicolumn{5}{p{11.6cm}|}{评阅意见主要从工作表现、能力水平、设计或论文质量三方面评价,
			各评分项目内涵见``指导教师评阅论文评定成绩标准"。} \\\hline
		\multicolumn{7}{c}{最终论文成绩=指导老师评阅成绩$\times 30\%$+交叉评阅成绩$\times 30\%$+答辩评定成绩$\times 40\%$}\\
		%\multicolumn{7}{c}{注:优(90分以上);良(80~89);中(70~79);及格(60 ~69);不及格(60以下)}\\
	\end{tabular}}
%=======================交叉评阅表===============================%
\pagestyle{empty}
\section*{}\label{table:jiaochapingyue}
%\addcontentsline{toc}{section}{交叉评阅表}
{\zihao{5}
	\begin{tabular}{|c|c|c|c|c|c|c|}
		\multicolumn{7}{c}{\includegraphics[height=0.8cm,angle=0]{preample/xishi}}\\
		\multicolumn{7}{c}{\zihao{-2}{\textbf{本科毕业论文(设计)交叉评阅表}}} \\\hline
		\multicolumn{2}{|c|}{论文(设计)题目}    & \multicolumn{5}{l|}{\hfill{\biaoti\cobiaoti\hfill}}                          \\\hline
		\multicolumn{2}{|c|}{学生姓名}        & \multicolumn{2}{c|}{\minzi}   & 学号     & \multicolumn{2}{c|}{\xuehao}      \\\hline
		\multicolumn{2}{|c|}{学院、专业年级}  & \multicolumn{5}{c|}{\school、\zhuanye 专业~\nianji}                          \\\hline
		\multicolumn{2}{|c|}{评阅人}          & \multicolumn{2}{c|}{\jiaocha} & 评阅时间 & \multicolumn{2}{c|}{\jpingyueriqi}\\\hline
		\multicolumn{1}{|p{.3cm}|}{\vspace{4.2cm}评阅意见} & \multicolumn{6}{c|}{
			\parbox[t][11cm][c]{14cm}{
				%具体评阅意见在此引入
				\jcomments
				%---------------------
			}
		} \\\hline
		\multicolumn{7}{|c|}{}\\[-8pt]
		\multicolumn{7}{|c|}{\raisebox{1ex}[0pt]{\zihao{4}{成绩评定}}}   \\\hline
		\multicolumn{2}{|c|}{评分项目}     & 得分     & 评分项目             & 得分      & 评分项目            & 得分          \\\hline
		%\multicolumn{2}{|c|}{选题(20)}    & \jxuanti & 能力与态度(40) & \jnengliyutaidu & 质量水平(40)& \jzhiliangsuiping \\\hline
		\multicolumn{2}{|c|}{选题价值(20)} & \jxuanti & 计算机应用与规范(15) & \jjisuanji & 设计或撰写水平(40) & \jzhuanxie    \\\hline
		\multicolumn{2}{|c|}{语言表达(10)} & \biaoda  & 文献整理与分析(15)   & \jwenxian  &                    &               \\\hline
		\multicolumn{2}{|c|}{评定成绩}     & \multicolumn{5}{c|}{\jpingdingdengji}                                             \\\hline
		\multicolumn{2}{|c|}{评阅人签名}   & \multicolumn{5}{c|}{ } \\\hline
		%\multicolumn{2}{|c|}{评分项目}                    & 项目得分 & 评分项目                                               & 项目得分  & 评分项目                                         & 项目得分  \\\hline
		%   \multicolumn{2}{|p{2.23cm}|}{选题指导思想(10)}    & \sixiang & \multicolumn{1}{p{2.8cm}|}{文献资料整理与分析能力(10)} & \zhengli  & \multicolumn{1}{p{2.5cm}|}{毕业论文撰写水平(30)} & \xiezuo   \\\hline
		%   \multicolumn{2}{|c|}{选题价值(10)}                & \jiazhi  & 外文运用能力(5)                                        & \yingwen  & 规范化程度(10)                                   & \guifandu \\\hline
		%   \multicolumn{2}{|c|}{选题难度(5)}                 & \nandu   & 语言运用能力(5)                                        & \yuyan    &                                                  &          \\\hline
		%   \multicolumn{2}{|p{2.23cm}|}{综合运用知识能力(10)}& \nengli  & \multicolumn{1}{p{2.8cm}|}{计算机运用能力(5)}          & \computer &                                                  &          \\\hline
		%   \multicolumn{2}{|c|}{\textbf{总分}}                   & \multicolumn{5}{c|}{\jiaochageifen}          \\\hline
		%   \multicolumn{2}{|c|}{评阅人签名}                  & \multicolumn{5}{l|}{ }                        \\\hline
		\multicolumn{2}{|c|}{\raisebox{-1.8ex}{备注}}     & \multicolumn{5}{p{11.8cm}|}{评语意见主要从选题质量、能力水平、设计或论文质量三方面评价,
			各评分项目内涵见``指导教师评阅论文评定成绩标准"。} \\\hline
		\multicolumn{7}{c}{最终论文成绩=指导老师评阅成绩$\times 30\%$+交叉评阅成绩$\times 30\%$+答辩评定成绩$\times 40\%$}\\
		%\multicolumn{7}{c}{注:优(90分以上);良(80~89);中(70~79);及格(60~69);不及格(60以下)}\\
	\end{tabular}}

%=======================答辩记录表===============================%
%\begin{comment}
\section*{}\label{table:dabianjilu}
%\addcontentsline{toc}{section}{答辩记录}
\pagestyle{empty}{\zihao{5}
	\begin{tabular}{|p{.3cm}|p{2.2cm}|p{5cm}|p{1cm}|p{4cm}|}
		\multicolumn{5}{c}{\includegraphics[height=0.8cm,angle=0]{preample/xishi}} \\
		\multicolumn{5}{c}{\zihao{-2}\textbf{本科毕业论文(设计)答辩记录}}    \\\hline
		\multicolumn{2}{|c|}{\hspace*{\fill}论文(设计)题目\hspace*{\fill}} & \multicolumn{3}{c|}{\biaoti\cobiaoti}  \\\hline
		\multicolumn{2}{|c|}{\hspace*{\fill}学生姓名\hspace*{\fill}} & \hspace*{\fill}\minzi\hspace*{\fill} & \hspace*{\fill}学号\hspace*{\fill} & \hspace*{\fill}\xuehao\hspace*{\fill} \\\hline
		\multicolumn{2}{|c|}{学院、专业年级}                         & \multicolumn{3}{c|}{\school、\zhuanye 专业~\nianji} \\\hline
		\multicolumn{1}{|p{.3cm}|}{\vspace{2.7cm} 答辩记录} & \multicolumn{4}{c|}{
			\parbox[t][8.3cm][c]{12.6cm}{
				%具体答辩记录在此引入
				%答辩记录
%具体答辩记录在下面填写
%----------------------
答辩记录在文件夹~{\tt biaoge} 中的文件~{\tt dabianjilu.tex} 内填写。
%\begin{enumerate}
%      \item 递推数列非线性向线性转化的例子是哪个?\\
%      ---没来得及回答
%      \item 什么是非线性递推数列?线性递推数列是怎么定义的?例子\\
%      答:非线性递推数列就是递推关系为分式型或者含有根号和次方的数列。形如$a_{n+1} = \frac{aa_n+b}{ca_n+d}$
%      类型的,线性递推数列就是含有$a_{n+1} = \lambda_1a_1+\lambda_2a_2+\cdots+\lambda_na_n$ 递推关系式的数列。一般类型有
%      $a_{n+1} = pa_n+q$.
%      \item 有哪些形容词可以修饰递推数列?\\
%      答:高阶。(可以用齐次非齐次、常系数变系数、几阶)
%      \item 你所写的关系式$a_{n+1} = pa_n+q$是几阶的? \\
%      答:一阶。
%\end{enumerate}

				%-------------------
		}} \\\hline
		\multicolumn{1}{|p{.3cm}|}{\vspace{1.1cm} 评审意见} & \multicolumn{4}{c|}{
			\parbox[t][5.6cm][c]{13.6cm}{
				%具体评审意见也可在此引入
				\dcomments
				%--------------------
		}} \\\hline
		%\multicolumn{2}{|c}{答辩得分:}          & \multicolumn{1}{l}{\dabian\;分×\,0.4 = \dabianR}  &  \multicolumn{2}{l|}{评阅人给分:\quad \jiaochageifen\;分×\,0.3 = \jiaochaR} \\
		%\multicolumn{2}{|c}{指导教师给分:}      & \multicolumn{3}{l|}{\jiaoshigeifen\;分×\,0.3 = \jiaoshiR}                                    \\
		%\multicolumn{5}{|c|}{}\\[-8pt]
		\multicolumn{2}{|c}{{答辩评定成绩:}}    &  \multicolumn{3}{l|}{{\dabiancj}}                \\\hline
		\multicolumn{2}{|c}{答辩小组组长签名}    & \multicolumn{3}{|c|}{\hspace{4cm}\nian\hspace{.2cm}年\hspace{.5cm} 月\hspace{.5cm} 日} \\\hline
		\multicolumn{2}{|c}{答辩委员会主席签名}  & \multicolumn{3}{|c|}{\hspace{4cm}\nian\hspace{.2cm}年\hspace{.5cm} 月\hspace{.5cm} 日} \\\hline
		\multicolumn{5}{p{14.8cm}}{{\textbf{说明}}: 评审意见应包括:论文写作在立题、阐述过程和所涉知识方面是否符合本科毕业论文的要求;答辩中对所提问题给予的回答是充分、不够充分或者无法回答;学生答辩是否认真,状态是否良好。}\\
		\multicolumn{5}{p{14.8cm}}{最终论文成绩=指导老师评阅成绩$\times 30\%$+交叉评阅成绩$\times 30\%$+答辩评定成绩$\times 40\%$。}
		%\multicolumn{5}{c}{最终论文成绩=指导老师评阅成绩$\times 30\%$+交叉评阅成绩$\times 30\%$+答辩评定成绩$\times 40\%$}\\
		%\multicolumn{5}{c}{注:优(90分以上);良(80~89);中(70~79);及格(60~69);不及格(60以下)}\\
	\end{tabular}%\label{endofThesis}
}
%\end{comment}


\end{document}
%%%%%%%%%%%%%%%%%%%%%%%%%%%%%%%%%%%%%%%%%%%%%%%%%%%%%%%%%%%%%%%%%%%
%%%%%%%%%%===========THE END OF PAPER============%%%%%%%%%%%%%%%%%%
%%%%%%%%%%%%%%%%%%%%%%%%%%%%%%%%%%%%%%%%%%%%%%%%%%%%%%%%%%%%%%%%%%%
