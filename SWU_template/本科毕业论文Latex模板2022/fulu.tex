%\begin{appendix}
\section{Matlab 代码} \label{appendix:A}

下面是我写的一个~Matlab
代码,这个代码可以给出下面这个
1989年第30届国际数学奥林匹克(IMO)竞赛的一道试题的一个解答:\vspace{.5cm}

\noindent 求证:集合$\Set{1, 2,
\cdots,1989}$可以分为$117$个互不相交的子集
$$A_i, \qquad i = 1, 2, \cdots,117$$
使得
\begin{enumerate}
    \item 每个$A_i$都含有$17$个元素;
    \item 每个$A_i$中所有元素之和相同。
\end{enumerate}\vspace{.5cm}

这个代码的文件名为~{\tt equalsumpartition.m},运行时在~Matlab 命令窗口键入下面的命令
\begin{verbatim}
        equalsumpartition(1989,117)
\end{verbatim}
然后回车即可。
这里引入所用的命令是
\begin{verbatim}
       \lstinputlisting[language=Matlab]{codes/equalsumpartition.m}
\end{verbatim}
注意,因为文件~{\tt equalsumpartition.m} 存放在子文件夹~{\tt codes }中,所以引入时,文件名前面还必须加上路径名~{\tt codes/}。
%\textbf{\textcolor[rgb]{0.98,0.00,0.00}{ Input Matlab source:}}
\lstinputlisting[language=Matlab]{codes/equalsumpartition.m}
\section{Mathematica 代码}\label{appendix:B}
按第~\ref{sec:codeinludsion} 节介绍的方法直接引入~Mathematica 代码目前还有一些问题,一个临时解决办法是先将~Mathematica 代码另存为.txt 文件,然后用命令
\begin{verbatim}
        \lstinputlisting[language=Mathematica]{myfile.txt}
\end{verbatim}
引入。%注意,这里和第~\ref{sec:codeinludsion} 节介绍的方法的不同之处是没有了\verb|[language=代码语言]| 这一部分。
下面就是一个按这种方法引入的一个~Mathematica 代码:{\tt primeSieve.txt},这里用的命令是
\begin{verbatim}
       \lstinputlisting[language=Mathematica]{codes/primeSieve.txt}
\end{verbatim}
注意,文件~{\tt primeSieve.txt} 也存放在子文件夹~{\tt codes} 中。
这个代码实现了著名的~Eratosthenes 筛法。对一个任意给定的正整数$n$,通过~Eratosthenes 筛法,可以找出所有不超过$n$的素数。本代码首先在$[50,200]$中随机地选一个数作为$n$(如果要用其它的数,可通过调整代码中赋给的~{\tt m} 和~{\tt M} 的值来实现),然后给出所有不超过$n$的素数。如果要查看中间结果,只需将一个非$0$ 的值赋给变量$d$,然后运行程序即可。
\lstinputlisting[language=Mathematica]{codes/primeSieve.txt}
\section{R 代码}\label{appendix:C}
此例子由彭作祥教授提供,是$\mathbf{R}$软件下编写的程序,为寿险精算中的一个简单例子。计算
在每年末偿还金额中的利息和本金金额。这个代码的文件名为~{\tt
example.R},在$\mathbf{R}$中打开运行即可。引入命令是
\begin{verbatim}
       \lstinputlisting[language=R]{codes/example.R}
\end{verbatim}
\lstinputlisting[language=R]{codes/example.R}
\section{DES C++ 代码}\label{appendix:D}
下面是一个文件名为~{\tt mytestdes.cpp} 的~C++ 代码,引入命令是
\begin{verbatim}
       \lstinputlisting[language={[ANSI]C++}]{codes/mytestdes.cpp}
\end{verbatim}
这个代码实现了著名的对称密码算法~DES。
%\textbf{\textcolor[rgb]{0.98,0.00,0.00}{ Input Matlab source:}}
\lstinputlisting[language={[ANSI]C++}]{codes/mytestdes.cpp}
%\end{appendix}
